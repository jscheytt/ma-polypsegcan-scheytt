Kolorektale Karzinome haben eine hohe Sterblichkeitsrate, wenn sie spät entdeckt werden.
Eine frühzeitige Entfernung von bösartigen Polypen im Magen-Darm-Trakt, die deren Vorstufen bilden, bietet jedoch hohe Überlebenschancen.
Bei Darmspiegelungen werden gerade kleine Polypen aber recht häufig übersehen.
Zuverlässige bildverarbeitende Systeme, die Polypen in einem Koloskopie-Frame nicht nur detektieren, sondern pixelgenau segmentieren, könnten Ärzten bei Darmkrebs-Screenings helfen.

Diese Arbeit analysiert den aktuellen Stand der Segmentierung von Polypen im Gastrointestinaltrakt. Weiterführend wird untersucht, inwiefern die in letzter Zeit sehr erfolgreichen Methoden des Deep Learning hier Vorteile bieten.
