\chapter{Fazit}

Die Segmentierung von Kolorektalpolypen in koloskopischem Bildmaterial kann Ärzte bei der Entdeckung von potenziell bösartigen Adenokarzinomen unterstützen.
\acrlongpl{can} sind eine Weiterentwicklung der \glspl{gan} und werden für eine Bild-zu-Bild-Übersetzung trainiert.

In dieser Masterarbeit wurden sie zum ersten Mal für eine Polypensegmentierung im Speziellen und für eine binäre Segmentierung generell verwendet.
% TODO Platz in GIANA Kategorie eintragen
Beim Training auf einem niedrigauflösenden Dataset der \gls{giana} Sub-Challenge 2018 erzielt das Netz einen durchschnittlichen \gls{iou} von 0,3681 auf den Testdaten und erreicht damit in der Kategorie "Polyp Segmentation SD" den [x.] Platz von [y].

Die visuelle Qualität der Segmentierungen ist nicht optimal, da sie oft kleine False Positives enthält.
Viele Erkenntnisse deuten darauf hin, dass die Zielrepräsentation, die das Netz lernen soll, reichhaltiger als ein spärliches Binärbild sein müsste.
Dennoch ist das \gls{can} in der Lage, eine binäre Segmentierung zu lernen.
Zukünftige Verbesserungen können in einer Stabilisierung des Trainings durch neue Verlustfunktionen und Generator-Architekturen liegen.
