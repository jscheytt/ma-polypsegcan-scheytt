\chapter{Umsetzung}

Dieses Kapitel beschreibt die Umsetzung der gewählten Methodik mit den zur Verfügung stehenden Materialien.
Dazu wird auf das gewählte Framework, die Vorverarbeitung des Datasets und die durchgeführten Experimente eingegangen.



\section{TensorFlow}

Zur Entwicklung des \gls{can} wurde das Deep-Learning-Framework \emph{TensorFlow}~\cite{Abadi.2016} gewählt.
Zugrunde liegt TensorFlow das Paradigma, dass das Modell zuerst als Graph mit Entitäten und Operationen definiert und dann schrittweise trainiert wird.
Dies macht zwar die Programmierung etwas weniger intuitiv, sorgt aber dafür, dass viele Operationen im Voraus vereinfacht und teilweise auch parallelisiert werden können, sodass das Training insgesamt schneller abläuft.

\citeauthor{Isola.2017} stellen den originalen Code ihrer Publikation, der in Torch geschrieben ist, online zur Verfügung\footnote{\url{https://phillipi.github.io/pix2pix/}}.
Mehrere Portierungen auf andere Frameworks wurden bereits geschrieben, darunter auch mehrere für TensorFlow.
Gewählt wurde in dieser Arbeit die TensorFlow-Portierung von Christopher Hesse\footnote{\url{https://github.com/affinelayer/pix2pix-tensorflow}}, da diese von allen am besten dokumentiert ist.



\section{Vorverarbeitung}


