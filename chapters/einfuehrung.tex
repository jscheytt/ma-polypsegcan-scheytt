\chapter{Einführung}

Unter den 20 häufigsten Todesursachen weltweit sind drei Krebsarten vertreten, eine davon ist Darmkrebs~\cite{Lozano.2012}.
Sie hat die vierthöchste Sterblichkeitsrate von allen Krebsarten~\cite{Ferlay.2012} und liegt etwa 10~\% aller Krebstode zugrunde~\cite{Kumar.2005}.
Polypen in der Darmschleimhaut bilden dessen Vorstufe~\cite{Kumar.2005}.
Eine frühzeitige Entfernung bösartiger Kolorektalpolypen kann das Sterblichkeitsrisiko bis zur Hälfte reduzieren~\cite{Zauber.2012}, während die vollständige Resektion sämtlicher sowohl benigner als auch maligner Polypen unnötige Risiken wie Blutungen und Perforation aufweisen~\cite{Rex.2009}.

Regelmäßige Vorsorgeuntersuchungen ermöglichen es, das Wachstum von Kolorektalpolypen zu überwachen.
Hierbei wird in der Regel der Dickdarm im Rahmen einer Koloskopie endoskopisch untersucht~\cite{Kumar.2005} und von verdächtigen Polypen wird eine Gewebeprobe genommen, um diese anschließend histologisch auf ihre Gut- oder Bösartigkeit zu untersuchen.

Bei solchen Darmspiegelungen werden oftmals bis zu 28~\% aller Polypen übersehen~\cite{Leufkens.2012}.
Mit einer Markierung im endoskopischen Bild wie in \autoref{fig:highlight} könnte man Ärzten das Auffinden von Polypen erleichtern und die Fehlerrate verringern.
Grundlage dafür wäre eine automatische Lokalisierung von Polypen im koloskopischen Bild (s. \autoref{fig:segm}).

\begin{figure}[ht]
	\begin{subfigure}{.3\textwidth}
		\centering
		\includegraphics[width=.8\linewidth]{polyp129}
		\caption{}
		\label{fig:polyp}
	\end{subfigure}
	\begin{subfigure}{.3\textwidth}
		\centering
		\includegraphics[width=.8\linewidth]{segm129}
		\caption{}
		\label{fig:segm}
	\end{subfigure}
	\begin{subfigure}{.3\textwidth}
		\centering
		% TODO Create graphic
		\includegraphics[width=.8\linewidth]{segm129}
		\caption{}
		\label{fig:highlight}
	\end{subfigure}
	\caption{Kolorektalpolyp, dessen Segmentierung und Hervorhebung im Bild~\cite{Vazquez.2017}}
	\label{fig:polypseg}
\end{figure}

In der vorliegenden Masterarbeit wird zur Lösung dieses Lokalisierungsproblems ein Deep-Learning-Ansatz verwendet, der bisher noch nicht bei der Lokalisierung von Polypen eingesetzt wurde.
Dieser Ansatz, die \glspl{can} \cite{Isola.2017}, basiert auf den \glspl{gan} \cite{Goodfellow.2014}.

Dieses Kapitel erläutert die Problemstellung und stellt den Stand der Technik vor.
In den anschließenden Kapiteln erfolgt eine methodische Aufschlüsselung des Vorgehens in dieser Arbeit, die Umsetzung dieser Methodik, deren Ergebnisse und eine Analyse dieser Ergebnisse.



\section{Medizinischer Hintergrund}

\emph{Polypen} beginnen generell als leichte Erhebungen in der Schleimhaut, die in das Lumen des umgebenden Organs hineinragen~\cite{Kumar.2005}.
Anfangs wachsen sie \emph{sessil} ("stiellos") und sind nur als breite Kuppeln sichtbar, können aber im Lauf der Zeit einen Stiel entwickeln, sodass \emph{gestielte} Polypen in etwa pilzförmig aussehen.

Polypen lassen sich einteilen in \emph{neoplastisch} und \emph{nicht-neoplastisch}~\cite{Kumar.2005}.
Die häufigste neoplastische Polypenform ist das \emph{Adenom}, das sich im Laufe der Zeit zu Krebs entwickeln kann.
Alle anderen Polypenarten hingegen entwickeln sich fast ausschließlich gutartig.
Das Aussehen eines Polyps reicht in der Regel nicht aus zur Beurteilung der Malignität; dies ist bisher fast ausschließlich durch eine histologische Untersuchung des Polypengewebes möglich.
Allerdings steigt mit zunehmender Größe eines Polypen die Wahrscheinlichkeit, dass es sich um ein Karzinom handelt.

Karzinome, die sich aus Adenomen entwickeln, sogenannte \emph{Adenokarzinome}, sind die häufigste Form von bösartigen Tumoren im \gls{gi}~\cite{Kumar.2005}.
Im Kolorektalbereich, der vom allerletzten Abschnitt des Dünndarms bis zum Rektum reicht, tritt die stark überwiegende Mehrheit aller malignen Polypen im \gls{gi} auf.
Im Dünndarm hingegen treten sowohl benigne als auch maligne Polypen sehr selten auf; die Auftrittswahrscheinlichkeit sinkt weiter, je höher man im \gls{gi} aufsteigt.
In dieser Arbeit wird deshalb ausschließlich die Segmentierung von \emph{Kolorektalpolypen} betrachtet.



\section{Problemstellung und Motivation}

% Segmentierung, Lokalisierung ja - aber keine Klassifizierung gut/böse





\section{Manuelle Feature Selection}\label{sec:manuelle-feature-selection}



\subsection{Lokalisierung}



\subsection{Segmentierung}





\section{Deep Learning}\label{sec:deep-learning}



\subsection{Convolutional Neural Networks}



\subsection{Lokalisierung}



\subsection{Segmentierung}



\subsection{Generative Netze}





\section{Fazit}


