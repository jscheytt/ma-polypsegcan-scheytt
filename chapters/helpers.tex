%\begin{itemize}
%\setlength{\itemsep}{0pt}
%\setlength{\parsep}{0pt}
%	\item Seitengröße: DIN A5
%\end{itemize}
   

%\begin{table}
%\centering
%\caption{Tabellenbeschriftungen sollten über der Tabelle platziert werden}
%\begin{tabular}{|@{}c@{}|c|c|c|} \hline
%Graphics&Top&Inbetween&Bottom\\ \hline
%Tables & End& Last& First\\ \hline
%Figures & Good& Similar& Very Well\\ \hline
%\end{tabular}
%\end{table}

%Bitte berücksichtigen Sie bei der Verwendung von Bildern und Grafiken, dass nur in schwarz/weiß gedruckt wird. Der Einfachheit wegen sollten die Bilder bereits im Vorfeld in schwarz/weiß konvertiert sein, damit nicht durch eine vom Drucker angewandte schwarz/weiß Konvertierung, wichtige Bildinformationen verloren gehen.
%Fotos sollten eine Auflösung von 300 dpi haben, damit die Qualität für den Druck geeignet ist. Grafiken sollten 1200 dpi nutzen oder in einem vektorbasiertem Format wie bspw. EPS vorliegen. Ein Abbildungsverzeichnis ist nicht vorgesehen.
%\begin{figure}[h]
%\centering
%\psfig{file=test.eps, height=1in, width=1in,}
%\caption{Formatvorlage Abbildung}
%\label{test}
%\end{figure}
