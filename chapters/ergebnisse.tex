\chapter{Ergebnisse}

Dieses Kapitel stellt die Resultate der verschiedenen Experimente vor und zeigt auf, welche Design-Entscheidungen welche Auswirkungen haben.



\section{Erste Experimente und Debugging}

Der zur Verfügung stehende TensorFlow-Code bietet bereits einige Möglichkeiten, den Trainingsverlauf zu bewerten, da bestimmte Werte während des Trainings als Skalar ausgegeben und in Tensorboard als Verlauf angezeigt werden.
Zu diesen Werten gehören Mittelwerte der Gewichte und Bias-Werte jeder Schicht sowie deren Gradienten und drei Terme aus der Verlustfunktion:

\begin{enumerate}
	\item das \emph{Diskriminator-Loss}, der GAN-Term $ \mathcal{L}_c(D, G) $ der Gesamtverlustfunktion $ R^* $ (s. \autoref{eq:canlosswonoise})
	\item das \emph{Generator-Loss} $ D(\mathbf{x}, G(\mathbf{x})) $, die Bewertung des Diskriminators für den Output des Generators
	\item das \emph{L1-Loss}, der L1-Term $ \lambda \mathcal{L}_{L1}(G) $ der Gesamtverlustfunktion $ R^* $
\end{enumerate}

Ebenso gibt das Netz auch Bilder bzw. Aktivierungen bestimmter Schichten als Bild aus.
Unter diesen Bildausgaben sind neben Input, Target und Output auch die Aktivierungen der letzten Diskriminatorschicht für die Bewertung des realistischen Inputs $ D(\mathbf{x}, \mathbf{y}) $ und für die Bewertung des gefälschten Outputs $ D(\mathbf{x}, G(\mathbf{x})) $.
Diese werden im folgenden mit \emph{Real} und \emph{Fake} respektive bezeichnet.
Das Generator-Loss entspricht hierbei dem Mittelwert aller Aktivierungen von Fake.

Bei einem ersten Trainingslauf wurde das Netz für etwa 200 Epochen trainiert.
Demo-Datasets, die von den Autoren im Paper verwendet und online zur Verfügung gestellt werden, erzielen nach Trainingszeiten in dieser Größenordnung gute Ergebnisse.
In diesem Fall jedoch war der Output immer komplett schwarz, Real komplett weiß und Fake ebenfalls schwarz (s. \autoref{fig:outputsini}).
Letzteres zeigt, dass der Diskriminator maximal in der Lage ist, zwischen gefälschten und realistischen Bildern zu unterscheiden.
Die Verläufe der Losses des ersten Trainingslaufs in \autoref{fig:lossini} zeigen, dass das Diskriminator-Loss sofort auf Null fällt, das Generator-Loss ungewöhnlich große Sprünge aufweißt und das L1-Loss nach kürzester Zeit auf einem Plateau stagniert.

% TODO fig:outputsini Bilder von initial

% TODO fig:lossini Losses von initial und facades

Als Referenz ist in \autoref{fig:lossini} der Trainingsverlauf des Facades-Demo-Datasets aufgezeichnet:
Hier bewegt sich das Diskriminator-Loss auf den Wert 0.5 zu, der Verlauf des Generator-Loss ist sehr viel beständiger und das L1-Loss nimmt stetig ab.

Eine Debugging-Strategie, die \citeauthor{Goodfellow.2016} bei Deep Learning vorschlagen, ist das Trainieren mit einem sehr kleinen Dataset, um zu überprüfen, ob das Netz überhaupt in der Lage ist, aus den Trainingsdaten zu lernen~\cite{Goodfellow.2016}.
Zu diesem Zweck wurde ein Trainingslauf mit drei zufällig gewählten Datensätzen durchgeführt, und schon nach wenigen Trainingsschritten zeigte sich Erfolg (s. \autoref{fig:outputsminsamples} und \autoref{fig:lossminsamples}):

% TODO fig:outputsminsamples Bilder von min-samples

% TODO fig:lossminsamples Losses von min-samples, initial und facades angedeutet

Der Output ist nicht mehr leer, sondern nahe am Target, und Real und Fake sind sehr ähnlich.
Außerdem ist das Diskriminator-Loss nicht mehr auf Null, das Generator-Loss tendiert zu niedrigeren Werten und das L1-Loss verläuft monoton fallend.
Ein weiterer Trainingslauf mit den vollen Trainingsdaten und doppelter Trainingsdauer war dann erfolgreicher und führte zu ähnlich guten Ergebnissen wie der zweite Durchlauf mit minimaler Anzahl an Samples.



\section{Metriken}

Die \gls{giana} Challenge gibt zwei Metriken vor, um die Qualität der produzierten Segmentierungen zu bewerten: den Jaccard-Index~\cite{Jaccard.1901} und den Sørensen–Dice-Koeffizient~\cite{Srensen.1948,Dice.1945}.
Der Jaccard-Index ist auch bekannt als \gls{iou}, also Schnittmenge geteilt durch Vereinigungsmenge.
Die beiden Mengen sind in diesem Fall die Flächen mit Polypenpixeln im Target und im Output; die Schnittmenge ist die Überlappung beider Flächen.
Der \gls{iou} berechnet sich anhand der Anzahl wahr positiver (TP), falsch positiver (FP) und falsch negativer (FN) Samples bzw. anhand der Mengen A und B folgendermaßen:

\begin{equation}
J(A, B) = \frac{| A \cap B |}{| A \cup B |} = \frac{TP}{TP + FP + FN}
\end{equation}

Der Sørensen–Dice-Koeffizient ist identisch mit dem F1-Score, der das harmonische Mittel von Genauigkeit und Trefferquote ist.
Er ist folgendermaßen definiert:

\begin{equation}
SD(A, B) = \frac{2 \ | A \cap B |}{| A | + | B |} = \frac{2 \ TP}{2 \ TP + FP + FN}
\end{equation}

Die Werte beider Metriken liegen zwischen 0 und 1.
Im Gegensatz zum \gls{iou} erfüllt der F1-Score die Dreiecksungleichung nicht, wodurch nur der \gls{iou} sich als Distanzmaß eignet.
Beide Metriken sind insofern gleichwertig als dass sich der Wert einer Metrik durch den Wert der anderen berechnen lässt:

\begin{equation}
J = \frac{SD}{2 - SD} \ , \quad SD = \frac{2 \ J}{1 + J}
\end{equation}

Es ist direkt ersichtlich, dass der \gls{iou} nicht korrekt klassifizierte Pixel, also solche außerhalb der Schnittmenge, stärker bestraft als der F1-Score.



\section{Batch-Größen}


