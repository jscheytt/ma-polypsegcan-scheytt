Kolorektale Karzinome haben eine hohe Sterblichkeitsrate, wenn sie spät entdeckt werden.
Eine frühzeitige Entfernung von bösartigen Polypen im Magen-Darm-Trakt, die deren Vorstufen bilden, bietet jedoch hohe Überlebenschancen.
Bei Darmspiegelungen werden gerade kleine Polypen häufig übersehen.
Bildverarbeitende Systeme, die Polypen in einem Koloskopie-Frame nicht nur detektieren, sondern auch pixelgenau segmentieren, könnten Ärzten bei Darmkrebs-Screenings helfen.

Diese Masterarbeit setzt zum ersten Mal ein \acrlong{can} zur binären Segmentierung von Kolorektalpolypen ein.
Die optimale Batch-Größe sowie verschiedene Augmentierungstechniken werden evaluiert hinsichtlich der Stabilisierung des Trainings und einer Verbesserung der Ergebnisse.
Durch die Deaktivierung von Dropout zur Inferenzzeit und eine Batch-Größe von 32 werden Intersection-over-Union-Werte von bis zu 0,311 erzielt.
