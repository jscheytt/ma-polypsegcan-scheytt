\documentclass[
	12pt,
	a4paper,
	draft
]{report}

\usepackage[english,ngerman]{babel}
\addto{\captionsngerman}{\renewcommand{\abstractname}{Abstract}}
\usepackage[utf8]{inputenc}
\usepackage[a4paper, left=3.5cm, right=2.5cm, top=2.5cm, bottom=2cm]{geometry}
\usepackage[nottoc]{tocbibind}

\usepackage[
	colorlinks=true,
	citecolor=blue,
	linkcolor=black,
	filecolor=black,
	urlcolor=blue,
	linktoc=all
]{hyperref}

\usepackage{graphicx}
\graphicspath{ {img/} }
\usepackage{subcaption}

\usepackage[babel]{csquotes}
\MakeOuterQuote{"}

\usepackage[
	backend=biber,
	style=numeric-comp,
	minbibnames=10,
	maxbibnames=10,
	maxcitenames=2,
	giveninits=true,
	eprint=false,
]{biblatex}
\AtEveryBibitem{
	\clearlist{language}
	\clearfield{note}
	\clearfield{eventtitle}
	\clearfield{booksubtitle}
	\ifentrytype{misc}{
	}{
	\clearfield{url}
	}
}
\addbibresource{references.bib}
\DefineBibliographyStrings{german}{
	andothers = {et\addabbrvspace al\adddot},
	andmore = {et\addabbrvspace al\adddot},
}

% Erklärung
\usepackage{multicol}

% Math
\usepackage{amssymb}
\usepackage{amsmath}

% Acronyms
\usepackage[acronym]{glossaries}
\glstoctrue
\makeglossaries

\newacronym{cnn}{CNN}{Convolutional Neural Network}
\newacronym{rnn}{RNN}{Recurrent Neural Network}
\newacronym{wce}{WCE}{Wireless Capsule Endoscopy}
\newacronym{nbi}{NBI}{Narrow-Band Imaging}
\newacronym[plural=NN,firstplural=Neuronale Netze (NN)]{nn}{NN}{Neuronales Netz}
\newacronym{svm}{SVM}{Support Vector Machine}
\newacronym{fc}{FC}{Fully Connected}
\newacronym{fcn}{FCN}{Fully Convolutional Network}
\newacronym{gan}{GAN}{Generative Adversarial Network}
\newacronym{dcgan}{DCGAN}{Deep Convolutional GAN}
\newacronym{can}{CAN}{Conditional Adversarial Network}
\newacronym{gi}{GI}{Gastrointestinaltrakt}
\newacronym{pca}{PCA}{Principal Components Analysis}

\makeindex
