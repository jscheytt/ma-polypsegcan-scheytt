\documentclass[8pt,a5paper]{acm_proc_article-sp}
% fix umlauts
\usepackage[ngerman]{babel}
\usepackage[utf8]{inputenc} 
\usepackage[T1]{fontenc}  % Times new Roman
\usepackage{mathptmx}
\usepackage[a5paper, left=1.70cm, right=1.20cm, top=1.30cm, bottom=1.40cm]{geometry}
% balance columns. original from acm doesn't work
\usepackage{balance}
% colors
\usepackage[compact]{titlesec}
\titlespacing{\section}{0pt}{*0.7}{*0.5}

\usepackage[usenames,dvipsnames]{xcolor}
\usepackage{csquotes}
\MakeOuterQuote{"}

\usepackage[
	backend=biber,
	style=trad-abbrv,
	minbibnames=10,
	maxbibnames=10,
	giveninits=true,
	eprint=false,
]{biblatex}
\AtEveryBibitem{
	\clearlist{language}
	\clearfield{note}
	\clearfield{eventtitle}
	\clearfield{booksubtitle}
	\ifentrytype{misc}{
	}{
	\clearfield{url}
	}
}
\addbibresource{references.bib}

% automatic crosslinks
\usepackage[
	colorlinks=true,
	citecolor=black,
	linkcolor=black,
	filecolor=black,
	urlcolor=black,
]{hyperref}
%\hypersetup{colorlinks=true,citecolor=black,linkcolor=black,filecolor=black,urlcolor=black,pagebackref=true}

%glossaries
%\usepackage{makeidx}
%\makeindex
\usepackage[acronym,nomain,nonumberlist]{glossaries}

\newacronym{cnn}{CNN}{Convolutional Neural Network}
\newacronym{rnn}{RNN}{Recurrent Neural Network}
\newacronym{wce}{WCE}{Wireless Capsule Endoscopy}
\newacronym{nbi}{NBI}{Narrow-Band Imaging}
\newacronym[plural=NN,firstplural=Neuronale Netze (NN)]{nn}{NN}{Neuronales Netz}
\newacronym{svm}{SVM}{Support Vector Machine}
\newacronym{fc}{FC}{Fully Connected}
\newacronym{fcn}{FCN}{Fully Convolutional Network}
\newacronym{gan}{GAN}{Generative Adversarial Network}
\newacronym{dcgan}{DCGAN}{Deep Convolutional GAN}
\newacronym{can}{CAN}{Conditional Adversarial Networks}

\glsdisablehyper

\makeglossaries
\makeindex

%\newcommand{\glspl}[1]{{#1}}

% http://en.wikibooks.org/wiki/LaTeX/Glossary
% http://mirror.informatik.uni-mannheim.de/pub/mirrors/tex-archive/macros/latex/contrib/glossaries/glossariesbegin.pdf

% Definitionsliste
\newcommand{\defitem}[1]{\item[#1]\phantomsection\label{#1}\hfill\\} 
\newcommand{\defref}[1]{\hyperref[#1]{#1}} 
\newcommand{\rem}[1]{}
% Marking colors
\definecolor{todo}{rgb}{1,0.2,0.2}
\definecolor{reconsider}{rgb}{0.6,0.6,0.3}
\newcommand{\todo}[1]{{\color{todo} #1}}
\newcommand{\reconsider}[1]{{\color{reconsider} #1}}

\graphicspath{ {img/} }
